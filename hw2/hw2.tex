\documentclass{3330hw}
\hwnum{2}
\due{January 7th, 2013}

\begin{document}
\maketitle
	
\begin{enumerate}
\item One might choose to use the add instruction as opposed to lw as the addi instruction accepts immediate values - that is, constants do not necessarly have to be stored in a register before you add them to another register. More importantly, because of its structure type (``Type I"), 16 bits are available to process the offset from a base memory address. With lw, there are only five available bits. This limits the maximum offset value to 32, which is low enough to frequently cause problems. 

\item Loading the array into registers...
\begin{lstlisting}
; lw is used in this case as the offset value does not exceed 32

lw $t0 0($s3)
lw $t1 4($s3)
lw $t2 8($s3)
lw $t3 12($s3)
lw $t4 16($s3)
lw $t5 20($s3)
lw $t6 24($s3)
\end{lstlisting}
Storing the array in reverse order...
\begin{lstlisting}
sw $t6 0($s3)
sw $t5 4($s3)
sw $t4 8($s3)
sw $t3 12($s3)
sw $t2 16($s3)
sw $t1 20($s3)
sw $t0 24($s3)
\end{lstlisting}


\item Occasionally computer programs will contain more variables than the processor has registers. As such, the compiler will attempt to put less frequently used variables into memory instead. This allows more frequently accessed variables to benefit from the faster speed of registers.

\item MIPS pseudocode is used for this problem (no corresponding registers were given)
\begin{lstlisting}
sub h, h, 5
add f, g, h
\end{lstlisting}

\item Equivalent C instructions:
\begin{lstlisting}[language=c]
f = g + h + i;
\end{lstlisting}

\item~
\begin{lstlisting}
sub $t0, $s3, $s4
sll $t0, t0, 2		
add $t1, $s7, $t0   
lw $t2, 0($t1)      
sw $t2, 32($s6)     
\end{lstlisting}


\item ~
\begin{multline*}
\text{0xabcdef12 } = 10\times16^7 + 11\times16^6 + 12\times16^5 + 13\times16^4 +\\ 14\times16^3 + 15\times16^2 + 1\times16^1 + 2\times16^0 = 2882400018
\end{multline*}

\item Equivilant C instructions:
\begin{lstlisting}[language=c]
A[0] = A[1];
f = A[1] + A[0];
\end{lstlisting}

\item The instruction is an add instruction. It executes the following code: 
\begin{lstlisting}
add $s0, $s0, $s0
\end{lstlisting}

\item t2 has a value of 3

\item ~
\begin{lstlisting}
fib:
	slti $t0, $a0, 2	;continue statements 
	beq $t0, $zero, cont
	addi $v0, $zero, 1	;else make the return value 1 
	jr $ra

cont:									 
	;open up space in the stack
	addi $sp, $sp, -16  
	sw $ra, 0($sp)		
	sw $a0, 4($sp)
 
	; getting n-1
	addi $a0, $a0, -1   
	jal fib			    
	sw $v0, 8($sp)	    
	lw $a0, 4($sp)	    
	
	;getting n-2
	addi $a0, $a0, -2   
	jal fib
	sw $v0, 12($sp)	    
	
	;cleaning up the stack
	lw $ra, 0($sp)		
	lw $t0, 8($sp)	    
	lw $t1, 12($sp)	   
	addi $sp, $sp, 16
	
	;final computation
	add $v0, $t0, $t1	
 
	jr $ra

\end{lstlisting}
\end{enumerate}

\end{document}

