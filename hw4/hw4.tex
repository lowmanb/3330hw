\documentclass[]{3330hw}

\hwnum{4}
\due{March 19th, 2014}

\begin{document}
\maketitle

\begin{itemize}
    \item [4.1.1] ~
		\begin{table}[h!]
		\centering
		\begin{tabular}{@{}ccccccc@{}}
		\toprule
		RegWrite & MemRead & ALUMUX & MemWrite & ALUOP & RegMux & Branch \\ \midrule
		1        & 1       & 1(lmm) & 0        & ADD   & 1(Mem) & 0      \\ \bottomrule
		\end{tabular}
		\end{table}	
	
	\item [4.1.2] The data memory and branch add units are the only blocks that are not used.

	\item [4.1.3] The Branch add block produces an output that is not used, and the Data Memory block does not produce an output.
	
	\item [4.2.1] The instruction uses the processsor instruction memory block, a register read port, and the register write port.

	\item [4.2.2] The ALU needs to also be able to do shifts.

	\item [4.2.3] Accordingling, the ALU control must then support the new shift operation in the ALU.

	\item [4.4.1] Becuase I-MEM has the largerst latency, the cycle time is 400ps.

	\item [4.4.2] The longest latency datapath goes through the instruction memory, sign extend / shift left 2, add to compute new instruction, and final mux. The total latency is then
	\[
		400 + 100 + 20 + 20 + 30 = 570 \un{ps}
	\]

	\item [4.4.3] Conditional branches also go through Registers, Mux and ALU
	\[
		400 + 200 + 120 + 30 + 30 = 780 \un{ps}
	\]
	
\end{itemize}

\end{document}
